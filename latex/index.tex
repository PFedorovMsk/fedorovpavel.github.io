\hypertarget{index_intro_sec}{}\section{Об Optimal Filtering}\label{index_intro_sec}
Основная задача состоит в предоставлении A\+PI для реализации алгоритмов нелинейной фильтрации, их тестирования и сравнения.

Кроме A\+PI имеются готовые реализации нескольких фильтров и задач, на которых они тестируются. Периодически, добавляются новые алгоритмы и задачи.\hypertarget{index_for_compile}{}\section{Что нужно для старта?}\label{index_for_compile}
\hypertarget{index_compilers}{}\subsection{Компиляторы}\label{index_compilers}
Компилятор должен поддерживать стандарт C++11 или более новый. Иначе код не скомпилируется.

Точно скомпилируется (проверено) при использовании \begin{DoxyVerb}  Gсс 4.8 (или новее);

  Clang 5.1 (или новее).
\end{DoxyVerb}


При использоании более старых версий нужно смотреть, поддерживают ли они стандарт C++11.

Другие компиляторы не проверялись.\hypertarget{index_gui}{}\subsection{Графика}\label{index_gui}
Графический интерфейс (G\+UI) написан на Qt. Требуется версия не старше 5.\+0. Рекомендуется Qt 5.\+6.\+1 или новее.

G\+UI -\/ вещь опциональная, вся \char`\"{}математика\char`\"{} написана только с использованием S\+TL и библиотеки Eigen. Так что код можно собрать и вовсе без установленного на машине Qt.\hypertarget{index_qt_bugs}{}\subsection{И ещё...}\label{index_qt_bugs}
Если вы пользуетесь Windows и собираетесь собрать проект с G\+UI не забывайте следующее\+:

Путь к папке с проектом (исходным кодом) не должен содержать кириллицы.

Расположения типа \begin{DoxyVerb}  "C:\Пользователи\...\optimal_filtering2",

  "C:\...\Рабочий стол\...\optimal_filtering2" 
\end{DoxyVerb}


и т.\+д. приведут к невозможности собрать проект -\/ баг Qt на Windows которому 100 лет и, похоже, не будет исправлен никогда =)

Примеры \char`\"{}хорошего\char`\"{} расположения проекта\+: \begin{DoxyVerb}  "C:\Workspace\optimal_filtering2", 

  "D:\optimal_filtering2" \end{DoxyVerb}
 